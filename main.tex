\documentclass{article}
\usepackage[utf8]{inputenc}

\title{ML1}
\author{Music Genre Classification}
\date{February 2016}

\begin{document}

\maketitle

\textbf{Abstract}

\section{Introduction}
Music Information Retrieval is a small but growing field, and music genre classification is an important subsection of this field. With the rise of internet services like Pandora, Spotify, and other recommendation systems, the ability for a computer to tag and associate songs has become ever more pressing. Music genre classification is a challenging problem as many songs can be classified under multiple genres. Furthermore, a song is a complex audio wave that must be translated into a digital vector before being classified. This transformation process takes a continuous wave and condenses it down to a vector set, creating some information loss inherent in the problem. In this work, we focus on testing various classification methods, and various features for classifying 30 second music clips into their appropriate genres. This work builds on the work of Tzanetakis and Cook, who published a paper titled "Automatic Musical Genre Classification of Audio Signals," that focuses on the same gtzan dataset that is used in this project. We do a deeper analysis of different classifiers using a similar set of features, analyze the effects of including different features in the classification task, and dissect which genres lead to the highest classification confusion. Through our experiments, we discovered that Blues and Rock are the hardest for our classifiers to isolate from other genres, and thus we report results that both include Blues and Rock as well as exclude those two labels and data subsets. With Blues and Rock, the highest classification accuracy we were able to achieve was 70\%. Without the Blues and Rock data subset, we were able to achieve a 80\% accuracy. 
\section{Related Work}

\section{Our Work}
\subsection{Description of data and data processing}

\subsection{Classification Methods}

The following 8 different classifiers from the SciKitLearn Python Libraries were used. 
\begin{enumerate}
    \item \textit{K-nearest-neighbors(KNN)} 
    \item \textit{Logistic Regression with $l_2$ penalty(LR)} using one-vs-rest scheme for the multi-class classification
    \item \textit{Chained PCA and Logistic Regression (PCALR)}
    \item \textit{Support Vector Machine (SVM)} with an exponential kernel function
    \item \textit{Naive Bayes classifier(NB)} using the Gaussian Naive Bayes algorithm
    \item \textit{Random Forests Classifier}
    \item \textit{Decision Trees} 
    \item \textit{AdaBoost}
\end{enumerate}

\subsection{Evalulation}

\section{Results}

\section{Discussion}
- Discussion of results, behaviour of classification methods on songs

- Summary and conclusion, including extensions
\section{Conclusion}

\section{References}
-Pedregosa F, Varoquaux G, Gramfort A, Michel V, Thirion B, et al. (2011) Scikit-learn: Machine
learning in Python. Journal of Machine Learning Research 12: 2825–2830

\end{document}
