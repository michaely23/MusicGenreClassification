\documentclass{article}
\usepackage[utf8]{inputenc}

\title{ML1}
\author{Music Genre Classification}
\date{February 2016}

\begin{document}

\maketitle

\textbf{Abstract}

\section{Introduction}
\subsection{Related Work}

\section{Our Work}
\subsection{Description of data and data processing}

\subsection{Classification Methods}

The following 8 different classifiers from the SciKitLearn Python Libraries were used. 
\begin{enumerate}
    \item \textit{K-nearest-neighbors(KNN)} 
    \item \textit{Logistic Regression with $l_2$ penalty(LR)} using one-vs-rest scheme for the multi-class classification
    \item \textit{Chained PCA and Logistic Regression (PCALR)}
    \item \textit{Support Vector Machine (SVM)} with an exponential kernel function
    \item \textit{Naive Bayes classifier(NB)} using the Gaussian Naive Bayes algorithm
    \item \textit{PCA + Logistic Regression}
    \item \textit{Random Forests Classifier}
    \item \textit{Decision Trees} 
    \item \textit{AdaBoost}
\end{enumerate}

\subsection{Evalulation}

\section{Results}

\section{Discussion and Conclusion}
- Discussion of results, behaviour of classification methods on songs

- Summary and conclusion, including extensions

\section{References}
-Pedregosa F, Varoquaux G, Gramfort A, Michel V, Thirion B, et al. (2011) Scikit-learn: Machine
learning in Python. Journal of Machine Learning Research 12: 2825–2830

\end{document}
